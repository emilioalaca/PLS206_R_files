\documentclass[letterpaper]{article}
\usepackage[utf8]{inputenc}
\parskip 7.2pt

\title{Homework 06 Key. PLS206 Fall 2013}
\author{Emilio A. Laca}

\usepackage{Sweave}
\begin{document}
\Sconcordance{concordance:HW06KeyBoots.tex:HW06KeyBoots.Rnw:%
1 7 1 1 0 23 1 1 3 2 0 1 5 4 0 2 1 4 0 1 3 2 1 1 2 41 0 1 2 2 1 1 2 1 0 %
1 1 22 0 1 1 8 0 5 1 4 0 1 3 13 1 1 2 1 0 1 1 22 0 1 1 1 3 2 0 1 14 13 %
0 2 2 46 0 1 3 2 1}


\maketitle


\section{Mixed effects models.}

\subsection{Determine which fixed and random effects to include in a model.}

You are studying the relationship between growth rate of salamanders and water temperature. Sixty salamanders are randomly selected from the population as "subjects" for the study. Three tanks are randomly assigned to each of three temperature treatments (15, 17, 19 and 21 C). Five salamanders are randomly selected for each tank. Individual salamander weight is measured every 10 days during 6 months. List the fixed and random effects you include in the model.

Response variable is the weight of individual salamanders.
Fixed effects: days*temperature, higher order response surface or a nonlinear equation on two variables.
Random effects: lizard ID because each lizard is measured more than once, and tank, because more than one lizard is in each tank. Both random effects can be in the intercept and slopes.

\section{Bootstrapping.}


Use the \verb!"xmpl_Boots_original.csv"! file to obtain BCa bootstrapped confidence intervals for the intercept and slopes for the dataset with collinearity.


\subsection{Report your code including full documentation of each line. [20]}

\begin{Schunk}
\begin{Sinput}
> # Collinear data
> datab <- read.csv("../Examples/xmpl_Boots_original.csv", header=TRUE)
> betasr<-function(data,i){ # create function that yields the statistic of interest
+   bootsample<-data[i,] # generate a boostrap sample
+   model<-lm(Yr~X1r+X2r, data=bootsample) # regression using bootstrap sample
+   coef(model) #extract bootstrapped betas from model
+ }
> library(boot)
> Col.boot<-boot(datab,betasr,R=2000) # perform 2000 boostrap samples and models and store result in "Col.boot."
> 
\end{Sinput}
\end{Schunk}

\subsection{Report the confidence intervals. [10]}

\begin{Schunk}
\begin{Sinput}
> lapply(1:3, function (x) {boot.ci(Col.boot, index=x, type="bca" )})
\end{Sinput}
\begin{Soutput}
[[1]]
BOOTSTRAP CONFIDENCE INTERVAL CALCULATIONS
Based on 2000 bootstrap replicates

CALL : 
boot.ci(boot.out = Col.boot, type = "bca", index = x)

Intervals : 
Level       BCa          
95%   (-0.5403, -0.1089 )  
Calculations and Intervals on Original Scale

[[2]]
BOOTSTRAP CONFIDENCE INTERVAL CALCULATIONS
Based on 2000 bootstrap replicates

CALL : 
boot.ci(boot.out = Col.boot, type = "bca", index = x)

Intervals : 
Level       BCa          
95%   (-2.7107,  1.8781 )  
Calculations and Intervals on Original Scale

[[3]]
BOOTSTRAP CONFIDENCE INTERVAL CALCULATIONS
Based on 2000 bootstrap replicates

CALL : 
boot.ci(boot.out = Col.boot, type = "bca", index = x)

Intervals : 
Level       BCa          
95%   (-1.6335,  3.1098 )  
Calculations and Intervals on Original Scale
\end{Soutput}
\end{Schunk}

\subsection{Calculate the confidence intervals using a regular lm() as usual (parametric CI based on the usual assumptions) and compare to the bootstrapping results. Report the parametric CI's. [20]}

\begin{Schunk}
\begin{Sinput}
> original.model<-lm(data=datab,Yr~X1r+X2r)
> summary(original.model)
\end{Sinput}
\begin{Soutput}
Call:
lm(formula = Yr ~ X1r + X2r, data = datab)

Residuals:
    Min      1Q  Median      3Q     Max 
-0.6558 -0.4366 -0.2670  0.2357  2.0635 

Coefficients:
            Estimate Std. Error t value Pr(>|t|)    
(Intercept)  -0.3725     0.1039  -3.584 0.000949 ***
X1r          -0.5369     1.1395  -0.471 0.640188    
X2r           0.9834     1.1507   0.855 0.398129    
---
Signif. codes:  0 ‘***’ 0.001 ‘**’ 0.01 ‘*’ 0.05 ‘.’ 0.1 ‘ ’ 1

Residual standard error: 0.6618 on 38 degrees of freedom
Multiple R-squared:  0.3311,	Adjusted R-squared:  0.2959 
F-statistic: 9.405 on 2 and 38 DF,  p-value: 0.0004806
\end{Soutput}
\begin{Sinput}
> confint(original.model)
\end{Sinput}
\begin{Soutput}
                 2.5 %     97.5 %
(Intercept) -0.5828552 -0.1620758
X1r         -2.8437436  1.7698639
X2r         -1.3461262  3.3129603
\end{Soutput}
\begin{Sinput}
> plot(density(Col.boot$t[,2]))
> xs <- seq(-5,5,0.05)
> lines(x=xs,y=dnorm(xs, mean=coef(original.model)[2], sd=sqrt(vcov(original.model)[2,2])), lty=2)
> abline(v=confint(original.model)[2,])
> abline(v=boot.ci(Col.boot, index=2, type="bca")$bca[4:5], lty=2)
> 
\end{Sinput}
\end{Schunk}

The bootstrapped CI is a little narrower and shifted upwards, indicating that for this sample the original estimator is slightly biased. However, the differences are minimal.


\subsection{What kind of bootstrapping would you use in the Spartina data set? [5]}

Use case resampling for this data because the variables are random variables. Values of soil properties are not manipulated nor known and fixed.

\subsection{What kind of bootstrapping would you use in the P fertilization data? [5]}

Predictor values were set manipulative and could be reproduced. If the model provides a good explanation of the data and has homogeneous variance, use residual resampling. 

\subsection{Use the spartina soil (not biomass) data and bootstrapping to calculate CI's for eigenvalues 1-3. Report your code and the CI. [30]}

\begin{Schunk}
\begin{Sinput}
> spart <- read.csv("../Examples/spartina.txt", header=TRUE)
> str(spart)
\end{Sinput}
\begin{Soutput}
'data.frame':	45 obs. of  17 variables:
 $ loc : Factor w/ 3 levels "OI","SI","SM": 2 2 1 2 1 3 1 3 3 3 ...
 $ typ : Factor w/ 3 levels "DVEG","SHRT",..: 2 1 3 2 1 1 2 1 1 2 ...
 $ bmss: int  236 396 984 392 676 1196 640 1764 824 504 ...
 $ sal : int  31 30 30 30 33 29 38 26 26 26 ...
 $ pH  : num  3.2 3.25 4.1 3.25 5 4.6 4.75 5.2 4.85 3.7 ...
 $ K   : num  661 647 554 536 1442 ...
 $ Na  : num  13010 17307 7886 15061 35184 ...
 $ Zn  : num  24.69 31.29 9.88 22.13 16.45 ...
 $ H2S : int  -640 -610 -540 -640 -610 -630 -580 -610 -620 -600 ...
 $ Eh7 : int  -314 -328 -294 -390 -290 -342 -252 -322 -336 -280 ...
 $ acid: num  8.14 6.26 4.66 7.06 2.34 4.9 2.62 3.6 3.72 6.58 ...
 $ P   : num  97.7 60.9 16.6 39.7 20.2 ...
 $ Ca  : num  1748 1694 1020 1339 2150 ...
 $ Mg  : num  3060 3019 1966 2808 5169 ...
 $ Mn  : num  57.9 52.8 31.5 56.3 14.3 ...
 $ Cu  : num  3.93 3.73 1.75 3.44 5.02 ...
 $ NH4 : num  211.1 102.2 65.9 179.8 59.5 ...
\end{Soutput}
\begin{Sinput}
> library(boot)
> # I am including code to bootstrap the loadings to warn users about the
> # need to correct for direction reversals in the PC's.
> spart.pca <- prcomp(spart[,4:17])
> pca4boot <- function(pca.0, data, i) { #corrected for PC sign reversals
+   data.i <- data[i,]
+   pca.i <- prcomp(data.i[,4:17], scale=TRUE)
+   evals <- pca.i$sdev[1:3]^2
+   # if the PC direction reversed with respect to the original one,
+   # the signs below become negative and are used to reverse the
+   # signs of teh scores back to the original.
+   chng.sign1 <- sign(cor(pca.0$rotation[,1],pca.i$rotation[,1]))
+   chng.sign2 <- sign(cor(pca.0$rotation[,2],pca.i$rotation[,2]))
+   load.Zn.pc1 <- chng.sign1*cor(data.i[,8],pca.i$x[,1])
+   load.Zn.pc2 <- chng.sign2*cor(data.i[,8],pca.i$x[,2])
+   result <- c(evals,load.Zn.pc1,load.Zn.pc2)
+   return(result)
+ }
> boot.pca <- boot(spart,pca4boot,1999, pca.0=spart.pca)
> lapply(1:3, function (x) {boot.ci(boot.pca, index=x, type="bca" )})
\end{Sinput}
\begin{Soutput}
[[1]]
BOOTSTRAP CONFIDENCE INTERVAL CALCULATIONS
Based on 1999 bootstrap replicates

CALL : 
boot.ci(boot.out = boot.pca, type = "bca", index = x)

Intervals : 
Level       BCa          
95%   ( 3.821,  5.441 )  
Calculations and Intervals on Original Scale
Warning : BCa Intervals used Extreme Quantiles
Some BCa intervals may be unstable

[[2]]
BOOTSTRAP CONFIDENCE INTERVAL CALCULATIONS
Based on 1999 bootstrap replicates

CALL : 
boot.ci(boot.out = boot.pca, type = "bca", index = x)

Intervals : 
Level       BCa          
95%   ( 3.071,  4.214 )  
Calculations and Intervals on Original Scale

[[3]]
BOOTSTRAP CONFIDENCE INTERVAL CALCULATIONS
Based on 1999 bootstrap replicates

CALL : 
boot.ci(boot.out = boot.pca, type = "bca", index = x)

Intervals : 
Level       BCa          
95%   ( 1.137,  1.921 )  
Calculations and Intervals on Original Scale
Some BCa intervals may be unstable
\end{Soutput}
\begin{Sinput}
> 
\end{Sinput}
\end{Schunk}


\end{document}
