\documentclass{article}

\usepackage{Sweave}
\begin{document}
\HeaderA{qq.plot}{Quantile-Comparison Plots}{qq.plot}
\methaliasA{qq.plot.default}{qq.plot}{qq.plot.default}
\methaliasA{qq.plot.glm}{qq.plot}{qq.plot.glm}
\methaliasA{qq.plot.lm}{qq.plot}{qq.plot.lm}
\aliasA{qqp}{qq.plot}{qqp}
\keyword{distribution}{qq.plot}
\keyword{univar}{qq.plot}
\keyword{regression}{qq.plot}
\begin{Description}\relax
Plots empirical quantiles of a variable, or of studentized residuals from
a linear model, against theoretical quantiles of a comparison distribution.
\end{Description}
\begin{Usage}
\begin{verbatim}
qq.plot(x, ...)

qqp(...)

## Default S3 method:
qq.plot(x, distribution="norm", ylab=deparse(substitute(x)),
        xlab=paste(distribution, "quantiles"), main=NULL, las=par("las"),
        envelope=.95, labels=FALSE, col=palette()[2], lwd=2, pch=1, cex=1,
        line=c("quartiles", "robust", "none"), ...)

## S3 method for class 'lm':
qq.plot(x, main=NULL, xlab=paste(distribution, "Quantiles"), 
  ylab=paste("Studentized Residuals(", deparse(substitute(x)), ")",
    sep = ""), 
  distribution=c("t", "norm"), line=c("quartiles", "robust", "none"), 
  las=par('las'), simulate=FALSE, envelope=0.95, labels=names(rstudent), 
  reps=100, col=palette()[2], lwd=2, pch=1, cex=1, ...)
\end{verbatim}
\end{Usage}
\begin{Arguments}
\begin{ldescription}
\item[\code{x}] vector of numeric values or \code{lm} object.
\item[\code{distribution}] root name of comparison distribution -- e.g., \code{norm} for the
normal distribution; \code{t} for the t-distribution.
\item[\code{ylab}] label for vertical (empirical quantiles) axis.
\item[\code{xlab}] label for horizontal (comparison quantiles) axis.
\item[\code{main}] label for plot.
\item[\code{envelope}] confidence level for point-wise confidence envelope, or 
\code{FALSE} for no envelope.
\item[\code{labels}] vector of point labels for interactive point identification,
or \code{FALSE} for no labels.
\item[\code{las}] if \code{0}, ticks labels are drawn parallel to the
axis; set to \code{1} for horizontal labels (see \code{\LinkA{par}{par}}).
\item[\code{col}] color for points and lines; the default is the \emph{second} entry
in the current color palette (see \code{\LinkA{palette}{palette}}
and \code{\LinkA{par}{par}}).
\item[\code{pch}] plotting character for points; default is \code{1} 
(a circle, see \code{\LinkA{par}{par}}).
\item[\code{cex}] factor for expanding the size of plotted symbols; the default is
\code{1}.
\item[\code{lwd}] line width; default is \code{2} (see \code{\LinkA{par}{par}}).
Confidence envelopes are drawn at half this line width.
\item[\code{line}] \code{"quartiles"} to pass a line through the quartile-pairs, or
\code{"robust"} for a robust-regression line; the latter uses the \code{rlm}
function in the \code{MASS} package. Specifying \code{line = "none"} suppresses the line.
\item[\code{simulate}] if \code{TRUE} calculate confidence envelope by parametric bootstrap;
for \code{lm} object only. The method is due to Atkinson (1985).
\item[\code{reps}] integer; number of bootstrap replications for confidence envelope.
\item[\code{...}] arguments such as \code{df} to be passed to the appropriate quantile function.
\end{ldescription}
\end{Arguments}
\begin{Details}\relax
Draws theoretical quantile-comparison plots for variables and for studentized residuals
from a linear model. A comparison line is drawn on the plot either through the quartiles
of the two distributions, or by robust regression. 

Any distribution for which quantile and
density functions exist in R (with prefixes \code{q} and \code{d}, respectively) may be used. 
Studentized residuals are plotted against the
appropriate t-distribution.

The function \code{qqp} is an abbreviation for \code{qq.plot}.
\end{Details}
\begin{Value}
\code{NULL}. These functions are used only for their side effect (to make a graph).
\end{Value}
\begin{Author}\relax
John Fox \email{jfox@mcmaster.ca}
\end{Author}
\begin{References}\relax
Fox, J. (1997)
\emph{Applied Regression, Linear Models, and Related Methods.} Sage.

Atkinson, A. C. (1985)
\emph{Plots, Transformations, and Regression.} Oxford.
\end{References}
\begin{SeeAlso}\relax
\code{\LinkA{qqplot}{qqplot}}, \code{\LinkA{qqnorm}{qqnorm}},
\code{\LinkA{qqline}{qqline}}
\end{SeeAlso}
\begin{Examples}
\begin{ExampleCode}
x<-rchisq(100, df=2)
qq.plot(x)
qq.plot(x, dist="chisq", df=2)

qq.plot(lm(interlocks~assets+sector+nation, data=Ornstein), sim=TRUE)
\end{ExampleCode}
\end{Examples}




\end{document}
